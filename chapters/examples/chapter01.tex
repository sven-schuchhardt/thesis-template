\chapter{Einleitung}
\label{ch:intro}
Lorem ipsum at nusquam appellantur his, labitur bonorum pri no \citep{dueck:trio}. His no decore nemore graecis. In eos meis nominavi, liber soluta vim cu. Sea commune suavitate interpretaris eu, vix eu libris efficiantur.

%
% Section: Motivation
%
\section{Hintergrund und Kontext}
\label{sec:intro:context}
\graffito{Note: The content of this chapter is just some dummy text. It is not a real language.}
Die Digitalisierung eröffnet vielfältige Möglichkeiten für das Gesundheitswesen, insbesondere hinsichtlich der Gesundheitsversorgung [1]. Sie fördert eine schnellere Kommunikation, die Bereitstellung und das Teilen von Patientendaten sowie die Vereinfachung von Abläufen [1].

In diesem Kontext hat der deutsche Bundestag das "Gesetz zur Beschleunigung der Digitalisierung des Gesundheitswesens" verabschiedet [1], das darauf abzielt, Lösungen im digitalen Bereich zu schaffen [1]. Ein integraler Bestandteil dieses Gesetzes und der Implementierung digitaler Lösungen ist die Einführung der elektronischen Patientenakte (ePA) [2]. Die ePA fungiert als digitaler Speicherort für Informationen und Befunde von medizinischen Untersuchungen und Behandlungen für gesetzlich versicherte Patienten [3]. Eine weitere Komponente des Gesetzes betrifft die Vertiefung der Integration digitaler Gesundheitsanwendungen in den Versorgungsprozess [2]. Digitale Gesundheitsanwendungen sind dabei digitale Medizinprodukte, die Patienten im Rahmen verschiedener Behandlungen unterstützen [5].

Mit dem Fortschreiten der Digitalisierung und der Weiterentwicklung der elektronischen Patientenakte schreibt das Gesetz, Digitale Gesundheitsanwendungen-Verordnung in Paragraf § 6a vor, dass digitale Gesundheitsanwendungen nun verpflichtend einen Export ihrer verarbeiteten Daten und Befunde in die ePA ermöglichen müssen [5].



%
% Section: Motivation
%
\section{Motivation}
\label{sec:intro:motivation}
\graffito{Note: The content of this chapter is just some dummy text. It is not a real language.}
Somit bringt die zunehmende Digitalisierung im Gesundheitswesen nicht nur innovative Lösungen, sondern auch neue Herausforderungen mit sich. Eine dieser Herausforderungen, die im Zuge neuer Gesetze an Bedeutung gewinnt, ist das Problem der Interoperabilität im Bereich digitaler Gesundheitsanwendungen. Die Integration verschiedener Systeme und die Gewährleistung einer reibungslosen Kommunikation zwischen ihnen sind entscheidend, um das volle Potenzial der digitalen Gesundheit auszuschöpfen.

Das Aufkommen neuer gesetzlicher Vorgaben stellt die Gesundheitsbranche vor die Aufgabe, bisher ungelöste Fragen im Zusammenhang mit der Interoperabilität zu adressieren. Dies schafft nicht nur die Notwendigkeit, technische Lösungen zu entwickeln, sondern bietet auch die Chance, Pionierarbeit zu leisten und als Leitfaden für zukünftige Entwicklungen zu dienen. Die Relevanz dieser Problematik erstreckt sich über aktuelle Anforderungen hinaus und prägt die Richtung, in die sich die digitale Gesundheitslandschaft bewegt.

Eine Schlüsselüberlegung dabei ist die Integration neuer Systeme in bereits bestehende Strukturen. Die Entwicklung von Lösungen muss nicht nur die neuesten technologischen Standards berücksichtigen, sondern auch die nahtlose Einbindung in bestehende Infrastrukturen sicherstellen. Die Umsetzung von Interoperabilität sollte nicht nur als kurzfristige Anpassung, sondern als langfristige strategische Entscheidung betrachtet werden, um nachhaltige Fortschritte in der digitalen Gesundheitsversorgung zu gewährleisten.

In dieser Arbeit werden wir uns mit der Herausforderung der Interoperabilität in digitalen Gesundheitsanwendungen auseinandersetzen, Lösungsansätze entwickeln und dabei insbesondere den Fokus auf die Integration in bestehende Systeme sowie die Umsetzung nach geforderten Standards legen. Dieser Forschungsansatz ist nicht nur von aktueller Relevanz, sondern bietet auch die Möglichkeit, einen Beitrag zur Gestaltung der zukünftigen digitalen Gesundheitslandschaft zu leisten.

%
% Section: Ziele
%
\section{Ziel der Arbeit}
\label{sec:intro:goal}

Exposé zur Bachelorarbeit:

Konzeptionierung und Implementierung einer Anwendung zur Erstellung von validierten medizinischen Informationsobjekten aus Datensätzen digitaler Gesundheitsanwendungen 

von: Sven Schuchhardt

Problemstellung:

Eine grundlegende Voraussetzung zur Weiterentwicklung des deutschen Gesundheitswesens ist das Vorantreiben der Digitalisierung. Aufgrund dessen hat der Bundestag im Jahre 2020 beschlossen, die elektronische Patientenakte einzuführen. Die elektronische Patientenakte soll einen digitalen Gesundheitsordner für Patienten darstellen, also einen Speicherort für digitale Gesundheitsdokumente, mit der Möglichkeit, Dokumente mit Ärzten, Krankenhäusern und weiteren Leistungserbringern zu teilen, um schnellere Einsicht von Befunden zu ermöglichen und Mehrfachuntersuchungen zu vermeiden. 

Somit existieren neue Auflagen für Systeme und Anwendungen im Gesundheitssystem. Betroffen davon sind unter anderem digitale Gesundheitsanwendungen.Digitale Gesundheitsanwendungen sind Anwendungen, die zur Unterstützung von Behandlungen und zur Krankheitserkennung von Leistungserbringern auf Rezept verschrieben werden können. Durch die Einführung der elektronischen Patientenakte muss nun auch Nutzern von digitalen Gesundheitsanwendungen ermöglicht werden, auf Wunsch die Daten in der elektronischen Patientenakte zu speichern. 

Da viele unterschiedlich aufgebaute Systeme im Gesundheitswesen existieren und diese in der Lage sein müssen, Daten aus der elektronischen Patientenakte zu lesen, bearbeiten oder zu speichern, ist ein interoperables Format der Informationen unabdingbar. Das bedeutet, dass auch die Datensätze einer digitalen Gesundheitsanwendung verpflichtend in ein interoperables Format gebracht werden müssen. 
Die Kassenärztliche Bundesvereinigung, die unter anderem daran arbeitet, die Interoperabilität zu gewährleisten, hat somit eine Struktur für digitale Daten im deutschen Gesundheitswesen festgelegt. Die Struktur der Datenobjekte zur Interoperabilität wird durch sogenannte medizinische Informationsobjekte definiert.

Zur Generierung dieser medizinischen Informationsobjekte existieren aktuell noch keine Dienste. Nun muss eine Anwendung konzeptioniert und implementiert werden, die dazu in der Lage ist, aus Datensätzen von beliebigen digitalen Gesundheitsanwendungen medizinische Informationsobjekte zu generieren.
Die Problemstellung dieser Arbeit beschäftigt sich damit, eine Anwendung zu konzeptionieren und zu implementieren, die in der Lage ist, medizinische Informationsobjekte aus den Datensätzen einer oder mehreren digitalen Gesundheitsanwendungen zu generieren. 
Wie muss eine Anwendung aufgebaut sein und wie kann eine Anwendung die Datensätze aus digitalen Gesundheitsanwendungen zu einem medizinischen Informationsobjekt transformieren und ein validiertes Ergebnis liefern, sind somit grundlegende Fragen.



Erkenntnisinteresse

Als Hersteller einer digitalen Gesundheitsanwendung gibt es durch die neu erlassenen Regularien die Pflicht, dem Nutzer die Möglichkeit zu bieten, seine Daten aus der digitalen Gesundheitsanwendung in der elektronischen Patientenakte zu speichern. Ein eigenes Konzept muss erarbeitet werden zur Umsetzung einer Anwendung zur Generierung von medizinischen Informationsobjekten. Durch die erst neu dazugekommenen Anforderungen existieren noch keine Dienste, die in der Lage sind, medizinische Informationsobjekte aus den Datensätzen einer digitalen Gesundheitsanwendung zu generieren. 

Die Konzeptionierung der Anwendung soll somit die aktuelle Lücke im Gesundheitswesen schließen, die sich mit dem Prozess der interoperablen Datenbenutzung beschäftigt. Somit muss einerseits eine Anforderungsanalyse erstellt werden. Des Weiteren müssen mehrere Herangehensweisen der Architektur betrachtet werden, ein Implementierungsansatz zur Erstellung von medizinischen Informationsobjekten nach festgelegten Kriterien in der Anforderungsanalyse erstellt und die verschiedenen Möglichkeiten der Validierung müssen verglichen werden.

Fragestellungen

Die zentrale Frage beschäftigt sich damit, wie eine Anwendung konzeptioniert und umgesetzt werden muss, um valide medizinische Informationsobjekte für digitale Gesundheitsanwendungen zu erstellen. 
Relevante Teilfragen für das Thema sind:
nach welchen Kriterien, die innerhalb der Anforderungsanalyse  herausgefunden wurden, muss die Anwendung konzeptioniert werden?
Wie muss die Anwendung bereitgestellt werden, um medizinische Informationsobjekte für verschiedene digitale Gesundheitsanwendungen zu erstellen?
Was muss beachtet werden zur Erstellung von medizinischen Informationsobjekten?
Wie können medizinische Informationsobjekte validiert werden?

Ziele und Hypothesen

folgende Hypothesen die überprüft werden sollen lauten:
Zur Architektur der Anwendung:
ein Micro Service bietet sich zur Umsetzung als eigenständige Anwendung an um für mehrere digitale Gesundheitsanwendungen medizinische Informationsobjekte zu generieren
Validierung: 
ein funktionierender Validierungsschritt existiert bei dem aktuellen Forschungsstand nicht

Somit müssen Anforderungen und Kriterien erstellt werden, auf deren Grundlage die weiteren Entscheidungen zur Konzeptionierung und Implementierung der Anwendung getroffen werden können. Ein weiteres Ziel ist es, für die Architektur der Anwendung nach den zuvor ausgewählten Kriterien geeignete Technologien zu finden.
Für das Vorgehen der Erstellung und Validierung von medizinischen Informationsobjekten müssen nach den aufgestellten Kriterien die relevanten Möglichkeiten verglichen und eine Lösung entwickelt werden. 

Stand der Forschung
		
Die Kassenärztliche Bundesvereinigung hat für die Interoperabilität von Daten für das deutsche Gesundheitswesen die Struktur von medizinischen Informationsobjekten festgelegt. Diese Strukturen geben genau vor, wie Objekte, die medizinische Informationen beinhalten, aufgebaut sind. Die Struktur der medizinischen Informationsobjekte beruht auf dem US-Standard FHIR. Dieser gibt klare Definitionen und Vorgaben zur Speicherung von Daten in einem interoperablen Standard. Darauf wurden medizinische Informationsobjekte aufgebaut und weiterentwickelt. 

Für die Datensätze der digitalen Gesundheitsanwendung gibt es noch keine klaren Vorgehensweisen, wie diese in medizinische Informationsobjekte transformiert werden. 
Zudem gibt es keine Empfehlungen oder Vorgaben, welche Technologien oder Standards für das Erstellen von medizinischen Informationsobjekten benutzt werden.

Der aktuelle Stand der Forschung im Bezug auf die Validierung bietet unterschiedliche Validierungsschritte an, ohne diese zu evaluieren. Der Standard FHIR bietet eine Validierungsengine an, die sich auf die Strukturen im amerikanischen Gesundheitswesen bezieht. Somit muss untersucht werden, wie nun die medizinischen Informationsobjekte für das deutsche Gesundheitswesen validiert werden können. 

Methode 

Um eine Aussage über die Konzeptionierung der Anwendung zu treffen, wird der aktuelle Stand der Forschung analysiert. Als Grundlage dazu dient die Literaturrecherche. Auf Grundlage einer ausführlichen Zusammenstellung der bisherigen Literatur wird zudem die Anforderungsanalyse erstellt und Kriterien aufgestellt. 

Spezifischer werden die Fragen zur geeigneten Architektur mit einer Literaturrecherche beantwortet. Das Erstellen und Validieren von medizinischen Informationsobjekten wird sowohl empirisch betrachtet als auch durch den vorhandenen Stand der Forschung unterstützt.

Der Stand der Forschung zur Validierung wird betrachtet und auf das eigene Beispiel der Datensätze einer digitalen Gesundheitsanwendung angewandt und implementiert. Die unterschiedlichen Herangehensweisen müssen nach der Umsetzung evaluiert werden. Daraus wird nach aufgestellten Kriterien in der Anforderungsanalyse eine Position bezogen, um zu entscheiden, welche Validierung in der Anwendung verwendet wird. 

Auch zur Erstellung von medizinischen Informationsobjekten wird eine Literaturrecherche durchgeführt. Das bildet die Grundlage, um Bezug auf den aktuellen Stand der Forschung zu nehmen. Von diesem Stand aus werden unterschiedliche Herangehensweisen getestet und evaluiert. Am Ende wird eine Position bezogen, um zu entscheiden, welche Herangehensweise in der Anwendung verwendet wird.

Die Wahl, empirisch vorzugehen und sich gleichzeitig auf die Literaturrecherche zu stützen, wird deshalb gewählt, da der Stand der Forschung keine klaren Vorgehensweisen definiert. 

Somit wird als empirisches Vorgehen die Anwendung mit den zuvor recherchierten unterschiedlichen Lösungsansätzen implementiert. Jeder Lösungsansatz wird auf der Grundlage der Implementierung und nach den zuvor aufgestellten Kriterien evaluiert, dazu Position bezogen und entschieden, welche Lösung sich eignet.

Provisorische Gliederung 

Einleitung 
Grundlagen
Regularien 
digitale Gesundheitsanwendung
Interoperabilität
medizinische Informationsobjekte 
Anforderungsanalyse
Konzeptionierung 
Implementierung
Zusammenfassung und Ausblick
	

%
% Section: Struktur der Arbeit
%
\section{Gliederung}
\label{sec:intro:structure}
Nulla fastidii ea ius, exerci suscipit instructior te nam, in ullum postulant quo. Congue quaestio philosophia his at, sea odio autem vulputate ex. Cu usu mucius iisque voluptua. Sit maiorum propriae at, ea cum \ac{API} primis intellegat. Hinc cotidieque reprehendunt eu nec. Autem timeam deleniti usu id, in nec nibh altera.

%
% Section:Test
%
\section{Test}
\label{sec:intro:structure}
Test test test tes tes tes tes tes tes tes tes tes ytor te nam, in ullum postulant quo. Congue quaestio philosophia his at, sea odio autem vulputate ex. Cu usu mucius iisque voluptua. Sit maiorum propriae at, ea cum \ac{API} primis intellegat. Hinc cotidieque reprehendunt eu nec. Autem timeam deleniti usu id, in nec nibh altera.
